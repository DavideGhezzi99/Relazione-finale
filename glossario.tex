\newglossaryentry{script}{
    name=Script,
    description={Programma o sequenza di istruzioni che viene interpretata o portata a termine da un altro programma (invece che dal processore come nei linguaggi compilati)}
}
\newglossaryentry{File System}{
    name=File System,
    description={Il file system è il sistema di archiviazione del sistema operativo. Ha la funzione di gestire lo spazio nelle memorie ausiliarie del computer (es. hard disk, disco fisso esterno, pen drive, dvd-rom, ecc...)}
}
\newglossaryentry{render}{
   name=Render,
   description={Il processo che permette di ottenere, a partire da un modello tridimensionale elaborato al computer, un'immagine artificiale molto realistica}
}
\newglossaryentry{Unity Registry}{
   name=Unity Registry,
   description={Sezione dell'Editor di Unity da cui si scaricano ed installano determinati pacchetti}
}
\newglossaryentry{Asset Store}{
   name=Asset Store,
   description={Sito Web da cui si scaricano ed installano i pacchetti Unity non presenti nello Unity Registry (https://assetstore.unity.com/)}
}
\newglossaryentry{avatar}{
   name=Avatar,
   description={In informatica, rappresentazione grafica e virtuale di un visitatore di sito web}
}
\newglossaryentry{networking}{
   name=Networking,
   description={In informatica, sistema di collegamento in rete di più elaboratori e utenti, comprendente le piattaforme, i sistemi operativi, i protocolli e le architetture di rete}
}
\newglossaryentry{database}{
   name=Database,
   description={In informatica, archivio di dati strutturato in modo da razionalizzare la gestione e l'aggiornamento delle informazioni e da permettere lo svolgimento di ricerche complesse}
}
\newglossaryentry{runtime}{
   name=Runtime System,
   description={Il runtime system di un programma (o di un linguaggio di programmazione) è l'insieme dell'hardware e del software necessario come piattaforma per l'esecuzione di quel programma (o dei programmi scritti in quel linguaggio)}
}